\subsection{Máy chủ: Quản lý và Xử lý Dữ liệu Trung tâm}
\label{sec:server_overview}

Hệ thống Python đóng vai trò trung tâm, tích hợp dữ liệu từ hai nguồn độc lập để phát hiện té ngã:
\begin{enumerate}
    \item \textbf{ESP32 (cảm biến chuyển động)}: Triển khai tại hiện trường, gửi trạng thái té ngã và tọa độ GPS qua giao thức MQTT.
    \item \textbf{Camera IP (xử lý thị giác)}: Chạy trên máy chủ hoặc máy tính chuyên dụng, thực hiện phát hiện người, theo dõi đối tượng, và ước lượng tư thế bằng mô hình AI.
\end{enumerate}

Hệ thống đồng bộ hóa và tích hợp dữ liệu từ hai nguồn này để đưa ra quyết định phát hiện té ngã chính xác, kích hoạt cảnh báo qua AMI/Telegram, và ghi sự kiện vào cơ sở dữ liệu.

\subsubsection{Kiến trúc Hệ thống}
\label{subsubsec:system_overview}

Hệ thống được thiết kế theo mô hình phân tầng và mô-đun hóa, như minh họa trong Hình~\ref{fig:system_architecture}, bao gồm:
\begin{itemize}
    \item \textbf{Lớp thu nhận dữ liệu}:
    \begin{itemize}
        \item \texttt{comm/}: Nhận dữ liệu từ ESP32 qua MQTT, quản lý Telegram Bot và AMI để gửi cảnh báo.
        \item \texttt{detection/}: Xử lý khung hình từ camera IP, thực hiện phát hiện người, theo dõi, và ước lượng tư thế.
    \end{itemize}
    \item \textbf{Lớp xử lý tổng hợp}: \texttt{fall/} và \texttt{processing/} tích hợp dữ liệu từ ESP32 và camera, sử dụng \texttt{DetectionProcessor} để đồng bộ và ra quyết định té ngã.
    \item \textbf{Lớp đầu ra}: \texttt{database/} ghi sự kiện vào \texttt{fall\_events.db}, \texttt{comm/} gửi cảnh báo qua Telegram và AMI.
\end{itemize}

\begin{table}[H]
\centering
\caption{Cấu trúc các mô-đun và vai trò}
\label{tab:server_modules}
\begin{tabular}{|l|l|p{7cm}|}
\hline
\textbf{Module} & \textbf{Lớp (Layer)} & \textbf{Vai trò và chức năng} \\
\hline
\texttt{comm/} & Input/Output & Nhận dữ liệu MQTT, quản lý Telegram Bot và AMI \\
\hline
\texttt{detection/} & Input & Xử lý video, phát hiện người, theo dõi, ước lượng tư thế \\
\hline
\texttt{fall/} & Processing & Thuật toán phát hiện té ngã từ ESP32 và camera \\
\hline
\texttt{processing/} & Processing & \texttt{DetectionProcessor} đồng bộ và tích hợp dữ liệu \\
\hline
\texttt{database/} & Storage & Ghi sự kiện vào \texttt{fall\_events.db} \\
\hline
\texttt{config/} & Support & Cấu hình hệ thống (MQTT, AI, database, AMI, Telegram) \\
\hline
\texttt{models/} & Support & Lưu trữ mô hình YOLOv8n và trọng số \\
\hline
\texttt{utils/} & Support & Công cụ vẽ skeleton, hỗ trợ debug \\
\hline
\texttt{tests/} & Testing & Script kiểm tra logic phát hiện té ngã \\
\hline
\texttt{main.py} & Entry point & Điều phối dữ liệu và ra quyết định \\
\hline
\end{tabular}
\end{table}

\subsubsection{Cấu trúc Thư mục Dự án}
\label{subsubsec:project_structure}

Cấu trúc thư mục dự án được tổ chức để phản ánh mô-đun hóa của hệ thống:

\begin{minted}[fontsize=\footnotesize, breaklines, frame=single, linenos]{text}
intergrate_fall/
├── comm/
│   ├── ami_trigger.py
│   ├── mqtt_client.py
│   └── telegram_bot.py
├── config/
│   └── config.py
├── database/
│   └── database_manager.py
├── detection/
│   ├── human_detector.py
│   ├── person_tracker.py
│   └── skeleton_tracker.py
├── fall/
│   └── fall_detector.py
├── processing/
│   └── detection_processor.py
├── utils/
│   └── draw_utils.py
├── models/
│   └── yolov8n.pt
├── tests/
│   └── test_fall.py
├── main.py
├── fall_events.db
\end{minted}

\subsubsection{Luồng Xử lý Máy chủ}
\label{subsubsec:server_flow}

Hệ thống xử lý dữ liệu theo luồng sau (xem Hình~\ref{fig:server_flow}):
\begin{enumerate}
    \item Thu nhận dữ liệu từ ESP32 (qua MQTT) và camera IP (qua xử lý thị giác).
    \item \texttt{DetectionProcessor} đồng bộ và tích hợp dữ liệu, xác định trạng thái té ngã.
    \item Ghi sự kiện té ngã vào \texttt{fall\_events.db} và gửi cảnh báo qua AMI/Telegram.
\end{enumerate}

\begin{figure}[H]
\centering
\includegraphics[width=0.85\textwidth]{figures/server_flow.pdf}
\caption{Luồng dữ liệu và xử lý: tích hợp ESP32 và Camera IP qua \texttt{DetectionProcessor}.}
\label{fig:server_flow}
\end{figure}

\begin{minted}[linenos, frame=lines, fontsize=\small, bgcolor=lightgray, breaklines]{python}
while system_is_running:
    frame = camera_thread.get_latest_frame()
    sensor_data = mqtt_thread.get_latest_data()
    
    detections = human_detector.process(frame)
    poses = pose_estimator.process(detections)
    
    for person_id, (box, landmarks) in enumerate(detections):
        await processor.handle_camera_data(frame, person_id, box, landmarks)
    
    if sensor_data:
        await processor.handle_mqtt_data(sensor_data)
\end{minted}

\subsubsection{Kết hợp Dữ liệu Đa phương thức}
\label{subsubsec:multi_input_fusion}

\texttt{DetectionProcessor} tích hợp dữ liệu từ ESP32 (MQTT JSON) và camera IP (AI), thực hiện:
\begin{itemize}
    \item Xác thực dữ liệu ESP32: kiểm tra \texttt{device\_id}, \texttt{fall\_detected}, và tọa độ GPS.
    \item Xử lý camera: Phát hiện người, ước lượng tư thế, sử dụng \texttt{FallDetector} để xác định té ngã dựa trên landmarks.
    \item Đồng bộ dữ liệu dựa trên timestamp, kiểm tra cooldown (5 phút) để tránh lặp cảnh báo.
    \item Gửi cảnh báo qua AMI và Telegram, với cơ chế thử lại khi gặp lỗi mạng.
\end{itemize}

\begin{minted}[linenos, frame=lines, fontsize=\small, bgcolor=lightgray, breaklines]{python}
async def handle_camera_data(self, frame: Optional[np.ndarray], person_id: int, box: list, landmarks: list):
    """Process camera frame, detect falls, and send alerts if needed."""
    entity_id = f"camera_person_{person_id}"
    detector = self._get_or_create_detector(entity_id)
    is_fall = detector.detect_fall(landmarks)

    if frame is not None and isinstance(frame, np.ndarray) and frame.size > 0:
        status = "fall" if is_fall else "normal"
        draw_person(frame, box, landmarks, entity_id, status)

    if is_fall and self._can_alert(entity_id):
        fall_event = self._create_fall_event("camera", entity_id, latitude=0, longitude=0, has_gps_fix=False)
        fall_id = await self._insert_fall_event_with_retry(fall_event)
        if fall_id is None:
            return

        alert_msg = f"⚠️ Fall detected by camera for {entity_id}. Event ID: {fall_id}"
        logger.info(alert_msg)
        await self._send_alerts(alert_msg, frame)
        self._update_last_alert(entity_id)
\end{minted}

\begin{minted}[linenos, frame=lines, fontsize=\small, bgcolor=lightgray, breaklines]{python}
async def handle_mqtt_data(self, mqtt_msg: Any, topic: str = None) -> None:
    """Process MQTT data from ESP32, validate, and handle fall alerts."""
    if not isinstance(mqtt_msg, dict):
        try:
            mqtt_msg = json.loads(mqtt_msg)
        except (json.JSONDecodeError, TypeError):
            logger.error("[MQTT] Failed to parse JSON payload")
            return

    device_id = mqtt_msg.get("device_id")
    fall_detected = mqtt_msg.get("fall_detected")
    if not device_id or fall_detected is not True:
        return

    latitude = mqtt_msg.get("latitude")
    longitude = mqtt_msg.get("longitude")
    has_gps_fix = mqtt_msg.get("has_gps_fix", False)

    if self._can_alert(device_id):
        fall_event = self._create_fall_event("esp32", device_id, latitude, longitude, has_gps_fix)
        fall_id = await self._insert_fall_event_with_retry(fall_event)
        if fall_id is None:
            return

        gps_info = f"{latitude}, {longitude}" if has_gps_fix and latitude is not None else "Unknown"
        alert_msg = f"🚨 Fall detected by device {device_id} at GPS: {gps_info}. Event ID: {fall_id}"
        await self._send_alerts(alert_msg, None)
        self._update_last_alert(device_id)
\end{minted}

\subsubsection{Thuật toán Phát hiện Té ngã}
\label{subsubsec:fall_detection_algorithm}

Thuật toán trong \texttt{FallDetector} phân tích chuyển động và góc thân người dựa trên landmarks từ camera và dữ liệu cảm biến từ ESP32, như minh họa trong Hình~\ref{fig:python_fall_diagram}:
\begin{itemize}
    \item \textbf{Kiểm tra landmarks}: Xác thực landmarks với độ tin cậy $\geq 0.5$, đặt lại bộ đếm nếu không hợp lệ.
    \item \textbf{Tính toán góc và vận tốc}:
        \begin{itemize}
            \item Góc thân người: Tính dựa trên vai và hông, so với phương dọc ($>60^\circ$) hoặc ngang ($>45^\circ$).
            \item Vận tốc: Tính dịch chuyển thân người giữa các frame ($>0.5$ đơn vị).
        \end{itemize}
    \item \textbf{Xác định trạng thái}:
        \begin{itemize}
            \item \textit{Falling}: Góc (dọc $>60^\circ$ hoặc ngang $>45^\circ$) và vận tốc ($>0.5$) vượt ngưỡng, kéo dài $\geq 5$ frame.
            \item \textit{Lying}: Góc (dọc $>70^\circ$ hoặc ngang $>55^\circ$) kéo dài $\geq 5$ frame.
        \end{itemize}
    \item \textbf{Ra quyết định}: Nếu trạng thái \textit{Falling} hoặc \textit{Lying} đạt ngưỡng thời gian (5 frame), kết luận té ngã và đặt lại bộ đếm.
\end{itemize}

\begin{figure}[H]
\centering
\includegraphics[width=0.9\textwidth]{figures/python_fall_diagram.pdf}
\caption{Lưu đồ thuật toán phát hiện té ngã trong Python.}
\label{fig:python_fall_diagram}
\end{figure}

\subsubsection{Lưu trữ và Cảnh báo}
\label{subsubsec:data_storage_alerts}

Sự kiện té ngã được ghi vào cơ sở dữ liệu SQLite \texttt{fall\_events.db}:

\begin{minted}[linenos, frame=lines, fontsize=\small, bgcolor=lightgray, breaklines]{sql}
CREATE TABLE fall_events (
    id INTEGER PRIMARY KEY AUTOINCREMENT,
    timestamp DATETIME DEFAULT CURRENT_TIMESTAMP,
    source TEXT NOT NULL,
    entity_id TEXT NOT NULL,
    fall_detected BOOLEAN NOT NULL,
    latitude REAL,
    longitude REAL,
    has_gps_fix BOOLEAN,
    alert_status INTEGER DEFAULT 0
);
\end{minted}

