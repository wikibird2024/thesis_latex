\section{Logic Phát hiện Té ngã}
\label{sec:detection_logic}

% Giới thiệu chung về logic (đã có)
...

\subsection{Logic Phát hiện Dựa trên Thị giác}
\label{subsec:vision_logic}

Mô tả chi tiết về cách module thị giác hoạt động, từ việc thu thập khung hình đến việc sử dụng MediaPipe Pose để trích xuất keypoints. Sau đó, bạn sẽ nói rằng quy trình này được minh họa trong Sơ đồ \ref{fig:processing_flow}.

% Dán đoạn code TikZ của bạn vào đây
\begin{figure}[h!] % Sử dụng h! để LaTeX cố gắng đặt hình ở đây
    \centering
    \begin{tikzpicture}
        % ... (Đoạn code TikZ của bạn) ...
    \end{tikzpicture}
    \caption{Sơ đồ quy trình xử lý khung hình trong MediaPipe Pose Estimation}
    \label{fig:processing_flow}
\end{figure}

% Tiếp tục trình bày về thuật toán phân tích keypoints (tỉ lệ chiều cao/rộng,...)
...

\subsection{Logic Phát hiện Dựa trên IMU}
...
