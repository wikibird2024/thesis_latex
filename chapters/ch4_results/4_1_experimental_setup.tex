
\section{Thiết lập thực nghiệm}
\label{sec:experimental_setup}
Luận văn sẽ trình bày kết quả hoạt động kết hợp các thành phần \textbf{phần cứng}, \textbf{phần mềm} trong \textbf{môi trường thử nghiệm} được sử dụng để triển khai hệ thống phát hiện té ngã.  
Mục tiêu là cung cấp một cái nhìn tổng thể về cách thức xây dựng hệ thống, nhằm đảm bảo quá trình đánh giá hiệu năng có cơ sở rõ ràng và có thể tái lập.  

Dữ liệu được thu thập từ cảm biến tại thiết bị đầu cuối, xử lý sơ bộ trên vi điều khiển ESP32 và truyền về máy chủ xử lý trung tâm thông qua giao thức MQTT hoặc kênh SMS khẩn cấp.  
Sơ đồ tổng quan của hệ thống thử nghiệm được minh họa trong Hình~\ref{fig:system_structure}.

\begin{figure}[H]
    \centering
    \includegraphics[width=0.95\textwidth]{figures/resuilt_structure_diagram.pdf}
    \caption{Kiến trúc tổng quan của hệ thống thử nghiệm phát hiện té ngã.}
    \label{fig:system_structure}
\end{figure}

\subsection{Môi trường thử nghiệm}
Các thử nghiệm được thực hiện trong môi trường trong nhà, với nhiều kịch bản té ngã và hoạt động sinh hoạt bình thường. Dữ liệu được ghi nhận để đánh giá:
\begin{itemize}
    \item \textbf{Độ chính xác}: Xác suất hệ thống phát hiện đúng té ngã.
    \item \textbf{Độ trễ}: Thời gian từ khi xảy ra sự kiện đến khi người dùng nhận cảnh báo.
    \item \textbf{Độ ổn định}: Khả năng vận hành liên tục và độ tin cậy trong quá trình thử nghiệm.
\end{itemize}

\subsection{Phần cứng và phần mềm sử dụng}
Bảng~\ref{tab:hardware_software_setup} liệt kê chi tiết các thành phần phần cứng và phần mềm được sử dụng trong hệ thống thử nghiệm:

\begin{table}[H]
\centering
\caption{Danh mục phần cứng và phần mềm sử dụng trong thử nghiệm}
\label{tab:hardware_software_setup}
\begin{tabular}{|l|p{10cm}|}
\hline
\textbf{Hạng mục} & \textbf{Mô tả} \\
\hline
Máy chủ thử nghiệm & Laptop CPU Intel Core i7-2630QM, GPU NVIDIA GeForce GT 525M, hệ điều hành Linux Mint \\
\hline
Phần mềm máy chủ & Asterisk 22.4, Python 3.10.13, môi trường thực thi script giám sát và điều khiển \\
\hline
Thiết bị cảm biến đeo & ESP32 tích hợp module SIM hỗ trợ kết nối Wi-Fi và mạng GSM \\
\hline
Camera phụ trợ & ESP32-S3 với camera tích hợp; sử dụng webcam máy tính thay thế khi có lỗi kết nối \\
\hline
Thiết bị di động & Điện thoại thông minh hỗ trợ GSM và kết nối Internet, cài đặt phần mềm SIP mã nguồn mở Linphone 6.0.17 để nhận cuộc gọi và tin nhắn \\
\hline
\end{tabular}
\end{table}
