
\section{Hạn chế và Hướng phát triển}
\label{sec:limitation_future_work}

Mặc dù hệ thống đã chứng minh được tính khả thi, song nghiên cứu vẫn tồn tại một số hạn chế nhất định:

\begin{itemize}
    \item \textbf{Quy mô thử nghiệm còn giới hạn}: Hệ thống mới dừng ở mức kiểm chứng nguyên mẫu, chưa triển khai trên phạm vi rộng hoặc trong điều kiện thực tế đa dạng.
    \item \textbf{Số lượng thiết bị cảm biến và kênh tiếp nhận thông tin còn hạn chế}: Hệ thống chưa đủ để kiểm tra tính ổn định trong môi trường nhiều người dùng.
    \item \textbf{Thuật toán nhận diện té ngã chưa hoàn toàn ổn định}: Vẫn tồn tại khả năng xảy ra cảnh báo sai  hoặc bỏ sót.
    \item \textbf{Thiết bị đeo chưa được tối ưu}: Kích thước còn cồng kềnh, thiếu các tính năng tiện ích quan trọng như nút bấm báo khẩn cấp.
    \item \textbf{Kết nối mạng 4G chưa hoạt động hoàn chỉnh}: Mặc dù đã cấu hình được các tham số mạng nhưng chức năng truyền dữ liệu qua mạng di động vẫn chưa khai thác được.
    \item \textbf{Định vị GPS còn hạn chế}: Thiết bị có thể lấy vị trí ngoài trời, nhưng trong quá trình kích hoạt cảnh báo thực tế chưa thu thập và gửi được vị trí thành công.
\end{itemize}

Để khắc phục các hạn chế nêu trên và nâng cao khả năng ứng dụng, một số hướng phát triển được đề xuất như sau:

\begin{enumerate}
    \item \textbf{Mở rộng quy mô thử nghiệm}: Triển khai hệ thống cho nhiều người dùng, trong các bối cảnh sinh hoạt và điều kiện môi trường khác nhau để đánh giá độ tin cậy.
    \item \textbf{Cải thiện thuật toán phát hiện té ngã}: Ứng dụng các kỹ thuật học máy hoặc học sâu nhằm tăng độ chính xác, giảm cảnh báo sai và bỏ sót.
    \item \textbf{Tối ưu thiết bị đeo}: Thiết kế mạch phần cứng nhỏ gọn hơn, nâng cao tính tiện dụng, bổ sung nút bấm khẩn cấp hoặc cảm biến bổ trợ.
    \item \textbf{Hoàn thiện kết nối truyền thông}: Tích hợp đầy đủ chức năng kết nối 4G để đảm bảo truyền dữ liệu theo thời gian thực mà không phụ thuộc vào Wi-Fi.
    \item \textbf{Nâng cao chức năng định vị}: Tối ưu việc lấy dữ liệu GPS trong cả điều kiện ngoài trời và trong nhà (kết hợp thêm định vị Wi-Fi hoặc công nghệ khác).
    \item \textbf{Tích hợp nền tảng IoT}: Kết nối với hệ thống giám sát từ xa, ứng dụng di động hoặc nền tảng đám mây để quản lý, lưu trữ và phân tích dữ liệu dài hạn.
\end{enumerate}
