
\section{Kết luận chung}
\label{sec:overall_conclusion}

Trong khuôn khổ luận văn, một hệ thống phát hiện và cảnh báo té ngã dựa trên thiết bị đeo cảm biến và xử lý phần mềm đã được xây dựng thành công ở mức nguyên mẫu. Hệ thống tích hợp các thành phần phần cứng (ESP32, cảm biến MPU6050, module SIM 4G--GPS, loa cảnh báo, LED chỉ thị) với các thành phần phần mềm (ESP-IDF, giao thức MQTT, server trung gian, bot Telegram), cho phép phát hiện sự kiện té ngã và gửi cảnh báo đến người dùng cuối. Kết quả thử nghiệm bước đầu cho thấy hệ thống hoạt động đúng theo mục tiêu thiết kế, với độ trễ đầu-cuối duy trì ở mức khả chấp nhận cho các ứng dụng cảnh báo khẩn cấp.  

Bên cạnh kết quả triển khai, quá trình nghiên cứu mang lại nhiều giá trị học thuật và thực tiễn. Về kiến thức, tác giả đã nắm vững hơn về các nguyên lý phát hiện té ngã dựa trên cảm biến, các giao thức truyền thông IoT, cũng như cách triển khai một hệ thống phân tán có tích hợp cơ sở dữ liệu và dịch vụ cảnh báo theo thời gian thực. Về kỹ năng, tác giả có thêm kinh nghiệm trong thiết kế phần mềm nhúng theo hướng module hoá, tổ chức cấu trúc dự án chuyên nghiệp, và triển khai thực nghiệm với nhiều công nghệ kết hợp.  

Đặc biệt, nghiên cứu này giúp làm rõ khoảng cách giữa mô hình thử nghiệm trong phòng lab và yêu cầu của một sản phẩm thương mại. Những khó khăn gặp phải, như việc tối ưu phần cứng đeo, độ chính xác của giải thuật, và độ ổn định của truyền thông, đều cho thấy các thách thức quan trọng cần giải quyết trong các giai đoạn phát triển tiếp theo.  

Tóm lại, luận văn đã không chỉ hiện thực hóa một hệ thống nguyên mẫu khả thi, mà còn tạo nền tảng tri thức và kinh nghiệm quý báu cho các nghiên cứu và phát triển ứng dụng IoT trong lĩnh vực y tế và chăm sóc sức khoẻ trong tương lai.
