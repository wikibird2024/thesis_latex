\section{Tóm tắt kết quả thực hiện hệ thống phát hiện và cảnh báo té ngã}
\label{sec:summary}

Phần tóm tắt sẽ tổng hợp những kết quả mà đề tài luận văn đã đạt được. 
Hệ thống \textit{FDAS} dựa trên công nghệ IoT đã được xây dựng và kiểm chứng thực nghiệm. 
Các nội dung chính bao gồm:  

\begin{itemize}
    \item \textbf{Xây dựng kiến trúc hệ thống}: gồm phần cứng (ESP32, MPU6050, mô-đun SIM4G--GPS, còi cảnh báo) và phần mềm (FreeRTOS, giao tiếp UART/I\textsuperscript{2}C, MQTT, dịch vụ cảnh báo qua Telegram, Python xử lý ảnh), đảm bảo khả năng mở rộng và thích ứng với nhiều bối cảnh.
    
    \item \textbf{Phát triển thuật toán phát hiện té ngã}: 
    \begin{itemize}
        \item Dữ liệu cảm biến (gia tốc, con quay hồi chuyển) được xử lý thời gian thực trên ESP32.
        \item Dữ liệu hình ảnh/video (RGB, IR, 3D) được phân tích bằng Python để trích xuất đặc trưng và ước lượng tư thế.
        \item Kết hợp dữ liệu đa phương thức (sensor fusion) nhằm tăng độ chính xác, giảm cảnh báo giả, mở rộng tầm giám sát và phù hợp với nhiều bối cảnh khác nhau.
    \end{itemize}
    
    \item \textbf{Triển khai cơ chế cảnh báo}: sử dụng buzzer, LED, tin nhắn SMS, dịch vụ MQTT--Telegram và SIP để thông tin đến người giám sát kịp thời, bảo đảm khả năng theo dõi liên tục.
    
    \item \textbf{Thực nghiệm và kiểm chứng}: đánh giá hệ thống trong nhiều môi trường thực tế, đo lường độ chính xác phát hiện, độ trễ cảnh báo và khả năng hoạt động ổn định.
\end{itemize}

Kết quả cho thấy FDAS có tiềm năng ứng dụng cao trong giám sát sức khỏe từ xa, đặc biệt đối với người cao tuổi và bệnh nhân cần theo dõi an toàn. 
Việc tích hợp xử lý ảnh cùng dữ liệu cảm biến giúp hệ thống vừa thông minh vừa đáp ứng yêu cầu thời gian thực, nâng cao độ tin cậy so với các giải pháp đơn phương.
