\section{Các Giao Thức Truyền Thông}
\label{sec:communication_protocols}

Phần này trình bày các giao thức truyền thông cốt lõi cho phép trao đổi dữ liệu và điều khiển trong hệ thống cảnh báo viễn thông nhúng IoT, bao gồm: Giao thức Khởi tạo Phiên (SIP), Message Queuing Telemetry Transport (MQTT), và JavaScript Object Notation (JSON).

\subsection{Giao Thức Khởi Tạo Phiên (SIP)}
\label{subsec:sip_protocol}

SIP là giao thức báo hiệu nền tảng của VoIP, dùng để khởi tạo, quản lý và kết thúc các phiên đa phương tiện qua mạng IP~\cite{sip_rfc3261}. SIP hỗ trợ gọi thoại, hội nghị video, nhắn tin tức thời và thông tin hiện diện. Nó hoạt động ở lớp ứng dụng của mô hình TCP/IP, sử dụng các phương thức dựa trên văn bản tương tự HTTP để đảm bảo tính linh hoạt và dễ mở rộng.

\subsubsection{Kiến Trúc và Luồng Thông Điệp}
\label{subsubsec:sip_architecture}

SIP hoạt động thông qua trao đổi thông điệp giữa \textit{User Agent Clients} (UACs) và \textit{User Agent Servers} (UASs), với sự hỗ trợ của máy chủ proxy, registrar và redirect server. Các thông điệp chính gồm:

\begin{itemize}
\item \textbf{INVITE}: Khởi tạo phiên hoặc cuộc gọi.
\item \textbf{ACK}: Xác nhận phản hồi thành công cho INVITE.
\item \textbf{BYE}: Kết thúc phiên.
\item \textbf{CANCEL}: Hủy yêu cầu đang xử lý.
\item \textbf{REGISTER}: Đăng ký thông tin thiết bị.
\item \textbf{OPTIONS}: Truy vấn khả năng của máy chủ.
\end{itemize}

SIP kết hợp với \textit{Session Description Protocol} (SDP) để định nghĩa tham số phiên (như codec, cổng) và \textit{Real-time Transport Protocol} (RTP) để truyền dữ liệu thoại/video thời gian thực. Ngoài ra, RTCP (RTP Control Protocol) được sử dụng để giám sát chất lượng truyền dẫn.

\subsubsection{SIP trong Hệ Thống Cảnh Báo Asterisk}
\label{subsubsec:sip_asterisk_integration}

Trong hệ thống cảnh báo dựa trên Asterisk, SIP kết nối điện thoại IP, softphone và PBX qua SIP trunk, mang lại lợi ích:

\begin{itemize}
\item \textbf{Quản lý tập trung}: Đồng nhất cấu hình và quản lý thiết bị.
\item \textbf{Tương thích cao}: Hỗ trợ đa dạng nền tảng và thiết bị.
\item \textbf{Chuẩn mở}: Tích hợp dễ dàng với hạ tầng viễn thông hiện có.
\item \textbf{Bảo mật}: Hỗ trợ mã hóa qua TLS và SRTP để bảo vệ dữ liệu.
\end{itemize}

Asterisk đóng vai trò như một SIP server, xử lý đăng ký và định tuyến cuộc gọi.

\subsubsection{Luồng Phân Phối Cảnh Báo}
\label{subsubsec:sip_alert_flows}

\paragraph{Đăng ký}
Thiết bị SIP gửi REGISTER tới Asterisk để xác thực và lưu thông tin liên lạc, thường kèm theo thông tin xác thực (username, password) trong header Authorization.

\paragraph{Thiết lập phiên}
Ứng dụng Linux ra lệnh cho Asterisk khởi tạo cuộc gọi qua AMI (lệnh \texttt{Originate}). Asterisk gửi INVITE, nhận phản hồi (180 Ringing, 200 OK) và xác nhận bằng ACK. Có thể dùng header \texttt{Alert-Info: ;info=alert-autoanswer} để tự động trả lời, giúp phân phối cảnh báo nhanh chóng mà không cần tương tác thủ công.

\paragraph{Trao đổi dữ liệu}
Dữ liệu cảnh báo truyền qua RTP. Nếu dùng TTS (Text-to-Speech), Asterisk tạo âm thanh từ văn bản và phát tới người nhận qua các codec như G.711 hoặc G.729.

\paragraph{Kết thúc phiên}
Một bên gửi BYE, bên kia phản hồi 200 OK để đóng phiên an toàn.

\subsubsection{Tích Hợp SMS qua SIP}
\label{subsubsec:sip_sms}

Có thể gửi SMS trong Asterisk bằng cách kích hoạt \texttt{textsupport=yes} trong \texttt{sip.conf} và định nghĩa logic xử lý trong \texttt{extensions.conf}. Ví dụ, sử dụng lệnh \texttt{MessageSend} để gửi tin nhắn SIP MESSAGE, tích hợp với các nhà cung cấp SMS gateway để chuyển tiếp sang mạng di động.

\subsection{Giao Thức Message Queuing Telemetry Transport (MQTT)}
\label{subsec:mqtt_protocol}

MQTT là giao thức nhẹ, tối ưu cho truyền thông M2M trong IoT, hoạt động trên TCP/IP với cơ chế kết nối lâu dài~\cite{mqtt_oasis_standard}. Nó được thiết kế cho các thiết bị có tài nguyên hạn chế, băng thông thấp và kết nối không ổn định.

\subsubsection{Kiến Trúc Publish/Subscribe}
\label{subsubsec:mqtt_pubsub}

MQTT dùng mô hình publish/subscribe qua broker trung tâm (như Mosquitto hoặc HiveMQ), mang lại:
\begin{itemize}
\item \textbf{Tách rời không gian}: Không cần biết địa chỉ IP của nhau.
\item \textbf{Tách rời thời gian}: Không yêu cầu kết nối đồng thời, hỗ trợ retained messages.
\item \textbf{Tách rời đồng bộ}: Truyền và nhận độc lập, giảm độ trễ.
\item \textbf{Bảo mật}: Hỗ trợ TLS, xác thực username/password, và ACL để kiểm soát truy cập topic.
\end{itemize}

Các lệnh chính: CONNECT, PUBLISH, SUBSCRIBE, UNSUBSCRIBE, DISCONNECT. Topic sử dụng cấu trúc phân cấp như \texttt{sensor/room/temperature}.

\subsubsection{Mức QoS}
\label{subsubsec:mqtt_qos}

\begin{itemize}
\item \textbf{QoS 0 (At most once)}: Gửi một lần, không xác nhận, phù hợp cho dữ liệu không quan trọng.
\item \textbf{QoS 1 (At least once)}: Gửi ít nhất một lần, có xác nhận (PUBACK), có thể trùng lặp.
\item \textbf{QoS 2 (Exactly once)}: Gửi đúng một lần qua bắt tay bốn bước (PUBREC, PUBREL, PUBCOMP), đảm bảo tin cậy cao.
\end{itemize}

Trong hệ thống IoT, QoS được chọn dựa trên mức độ quan trọng của dữ liệu cảnh báo.

\subsection{JavaScript Object Notation (JSON)}
\label{subsec:json_format}

JSON là định dạng dữ liệu nhẹ, dễ đọc, dùng rộng rãi trong IoT và viễn thông để trao đổi dữ liệu có cấu trúc~\cite{json_rfc8259}. Nó dựa trên cú pháp JavaScript nhưng độc lập ngôn ngữ.

\subsubsection{Ứng Dụng JSON}
\label{subsubsec:json_applications}

\begin{itemize}
\item Lưu cấu hình thiết bị và máy chủ.
\item Trao đổi dữ liệu cảm biến, cảnh báo giữa các thành phần hệ thống.
\item Tương thích đa nền tảng, dễ phân tích bằng các thư viện như json-c cho C, hoặc ArduinoJson cho ESP32.
\item Hỗ trợ tích hợp với MQTT payload hoặc SIP message body.
\end{itemize}

\subsubsection{Ví Dụ JSON}
\label{subsubsec:json_examples}


% --- JSON Example 1 ---
\begin{figure}[H]
\centering
\begin{minted}[frame=single, breaklines, bgcolor=lightgray, xleftmargin=10pt, linenos]{json}
{
  "network": {
    "ssid": "IoT_Network",
    "password": "MatKhauBaoMat123",
    "mqtt_broker": "192.168.1.100",
    "mqtt_port": 1883
  },
  "device": {
    "id": "ESP32_Sensor_01",
    "location": "Tang_May_A"
  }
}
\end{minted}
\caption{Cấu hình ESP32}
\label{fig:json_esp32_config}
\end{figure}

% --- JSON Example 2 ---
\begin{figure}[H]
\centering
\begin{minted}[frame=single, breaklines, bgcolor=lightgray, xleftmargin=10pt, linenos]{json}
{
  "device_id": "ESP32_Sensor_01",
  "timestamp": "2024-03-15T10:30:00Z",
  "readings": {
    "temperature": {
      "value": 87.5,
      "unit": "celsius",
      "status": "critical"
    }
  }
}
\end{minted}
\caption{Dữ liệu Cảm Biến}
\label{fig:json_sensor_data}
\end{figure}
