
\section{Cơ sở lý thuyết về phần cứng và cảm biến trong hệ thống phát hiện té ngã}

\subsection{Giới thiệu tổng quan}
Hệ thống phát hiện té ngã là một giải pháp toàn diện để giám sát và cảnh báo sự cố, tích hợp các thành phần cốt lõi bao gồm \textbf{thiết bị đeo người} hoặc \textbf{thiết bị gắn trong nhà} (chứa cảm biến và vi điều khiển), \textbf{module truyền thông}, và \textbf{máy chủ xử lý}. Các thiết bị này phối hợp để thu thập dữ liệu thời gian thực, xử lý và đưa ra cảnh báo kịp thời, tăng độ chính xác và độ tin cậy.

\subsection{Hệ thống nhúng và cảm biến}

\subsubsection{Vi điều khiển}
\textbf{ESP32} là trung tâm điều khiển của thiết bị nhúng. Với kiến trúc \textbf{lõi kép Xtensa LX6}, ESP32 có thể thực hiện đa nhiệm hiệu quả: một lõi xử lý dữ liệu cảm biến, lõi còn lại quản lý giao tiếp không dây như Wi-Fi và Bluetooth, giúp giảm độ trễ hệ thống. ESP32 hỗ trợ nhiều giao thức kết nối như \textbf{I2C} và \textbf{SPI}, tạo khả năng kết nối linh hoạt với các cảm biến.

\subsubsection{Cảm biến IMU}
IMU (Inertial Measurement Unit) là bộ phận quan trọng nhất để phát hiện chuyển động, bao gồm:
\begin{itemize}
    \item \textbf{Gia tốc kế}: Đo gia tốc tuyến tính $(a_x, a_y, a_z)$ dựa trên công nghệ MEMS để phát hiện sự thay đổi điện dung.
    \item \textbf{Con quay hồi chuyển}: Đo tốc độ góc $(\omega_x, \omega_y, \omega_z)$ dựa trên hiệu ứng Coriolis.
    \item \textbf{Từ kế}: Đo hướng từ trường để hiệu chỉnh sai số cho con quay hồi chuyển.
\end{itemize}

Dữ liệu thô từ IMU thường là giá trị số nguyên (ví dụ: từ -32768 đến 32767 với cảm biến 16-bit). Sau khi hiệu chuẩn và chuyển đổi, dữ liệu được biểu diễn dưới dạng vector 3D:
\[
\mathbf{a} = [a_x, a_y, a_z], \quad
\boldsymbol{\omega} = [\omega_x, \omega_y, \omega_z]
\]

\paragraph{Phát hiện té ngã dựa trên ngưỡng} 
Một sự kiện té ngã được xác định khi gia tốc tổng vượt ngưỡng $a_{th}$ và giảm mạnh xuống gần 1g (gia tốc trọng trường), đồng thời tốc độ góc thay đổi đột ngột:
\[
\|\mathbf{a}\| = \sqrt{a_x^2 + a_y^2 + a_z^2}
\]

\subsubsection{Module GPS}
Module GPS (ví dụ: u-blox NEO-6M) sử dụng nguyên lý \textbf{trilateration} để xác định vị trí bằng cách tính khoảng cách đến nhiều vệ tinh.  

Dữ liệu GPS được xuất ra dưới dạng chuỗi NMEA, ví dụ:
\begin{verbatim}
$GPRMC,081600.00,A,3404.7041778,N,104.25010901,W,0.20,1.26,190425,,,A*5B
\end{verbatim}
Các trường dữ liệu bao gồm thời gian, trạng thái, kinh độ, và vĩ độ. Dữ liệu này giúp xác định vị trí chính xác của người dùng để hỗ trợ dịch vụ cứu hộ.

\subsection{Hệ thống camera và máy chủ xử lý}

\subsubsection{Hệ thống camera}
Camera nhúng sử dụng \textbf{ESP32-S3} kết hợp cảm biến \textbf{OV5640} (5MP), cung cấp hình ảnh chất lượng cao. Dữ liệu hình ảnh được nén dưới định dạng \textbf{JPEG} và truyền qua Wi-Fi bằng các giao thức \textbf{HTTP} hoặc \textbf{RTSP}.

Luồng dữ liệu JPEG là một mảng byte liên tục, quan trọng để xác thực sự kiện té ngã dựa trên hình ảnh.

\subsubsection{Máy chủ xử lý}
Máy chủ chịu trách nhiệm xử lý dữ liệu phức tạp và đưa ra quyết định cuối cùng:
\begin{itemize}
    \item \textbf{Phần cứng}: Máy tính nhúng (ví dụ Raspberry Pi 4, NVIDIA Jetson Nano) hoặc máy chủ đám mây (AWS, Google Cloud) với CPU đa lõi, RAM lớn, và có thể có GPU để tăng tốc xử lý học sâu.
    \item \textbf{Phần mềm}: Hệ điều hành Linux với các nền tảng học máy như \textbf{TensorFlow} hoặc \textbf{PyTorch}.
    \begin{itemize}
        \item \textbf{Xử lý hình ảnh}: Thư viện \textbf{OpenCV} kết hợp mô hình học sâu (YOLO, MediaPipe) để phân tích video, phát hiện đối tượng và ước lượng tư thế.
        \item \textbf{Hệ thống liên lạc}: Công cụ như \textbf{Asterisk} để thực hiện cuộc gọi khẩn cấp tự động dựa trên dữ liệu cảnh báo từ thiết bị nhúng.
    \end{itemize}
\end{itemize}

Dữ liệu thô bao gồm luồng byte JPEG từ camera và chuỗi JSON hoặc gói MQTT từ ESP32. Máy chủ xử lý tổng hợp để xác minh té ngã và kích hoạt cảnh báo.

\subsection{Hệ thống truyền thông và liên kết tổng thể}

\subsubsection{Hệ thống truyền thông}
Module truyền thông kết nối thiết bị với dịch vụ bên ngoài:
\begin{itemize}
    \item \textbf{Module 4G/GPS/SMS (EC800K)}: Hỗ trợ kết nối 4G, GPS định vị và gửi/nhận SMS. Vi điều khiển giao tiếp bằng \textbf{AT commands}.
    \item \textbf{Wi-Fi (ESP32)}: Kết nối máy chủ hoặc thiết bị di động qua giao thức \textbf{MQTT} nhẹ, phù hợp cho IoT.
\end{itemize}

\subsubsection{Liên kết tổng thể}
\begin{enumerate}
    \item \textbf{Thu thập}: ESP32 thu thập dữ liệu từ IMU, GPS, và hình ảnh từ ESP32-S3.
    \item \textbf{Xử lý cục bộ}: ESP32 xử lý dữ liệu IMU bằng Kalman Filter để ước tính hướng và trạng thái, phát hiện té ngã ban đầu dựa trên ngưỡng.
    \item \textbf{Truyền dữ liệu}: ESP32 gửi dữ liệu JSON/MQTT và luồng JPEG đến máy chủ qua Wi-Fi hoặc EC800K.
    \item \textbf{Xử lý trên máy chủ}: Máy chủ Linux phân tích hình ảnh, xác minh té ngã và chuẩn bị cuộc gọi tự động.
    \item \textbf{Cảnh báo}: Sau khi xác nhận, máy chủ ra lệnh ESP32 kích hoạt EC800K gửi SMS, gọi điện hoặc thông báo đến điện thoại giám sát. Hệ thống đảm bảo dự phòng, chuyển sang mạng di động nếu Wi-Fi mất kết nối.
\end{enumerate}
