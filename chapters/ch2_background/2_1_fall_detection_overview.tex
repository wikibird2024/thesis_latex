\section{Tổng quan về các phương pháp phát hiện té ngã}

% Introducing the importance of fall detection and its global context
Phát hiện té ngã là một lĩnh vực nghiên cứu trọng yếu, đặc biệt trong bối cảnh dân số già hóa đang gia tăng trên toàn cầu. Theo Tổ chức Y tế Thế giới (WHO), té ngã là nguyên nhân đứng thứ hai gây tử vong do thương tích không cố ý trên toàn thế giới, với khoảng 684.000 ca tử vong và hàng triệu ca chấn thương mỗi năm~\cite{who2021}. Các hệ thống phát hiện té ngã tự động không chỉ giúp cảnh báo kịp thời mà còn hỗ trợ cứu sống, giảm thiểu thương tích và cải thiện chất lượng chăm sóc sức khỏe. Các phương pháp hiện nay được phân loại thành bốn nhóm chính: cảm biến đeo được, cảm biến môi trường, thị giác máy tính và kết hợp đa phương thức, mỗi nhóm đều có ưu và nhược điểm riêng biệt.

% Subsection for wearable sensor-based methods
\subsection{Phương pháp dựa trên cảm biến đeo được}

Phương pháp này sử dụng các thiết bị đeo trên cơ thể để thu thập dữ liệu chuyển động, được ứng dụng rộng rãi nhờ tính linh hoạt và chi phí thấp.

\begin{itemize}
    \item \textbf{Cơ chế hoạt động:} Các thiết bị đeo tích hợp cảm biến quán tính như \textbf{gia tốc kế} (đo gia tốc tuyến tính), \textbf{con quay hồi chuyển} (đo tốc độ góc) và đôi khi \textbf{từ kế} (đo định hướng). Dữ liệu được phân tích thời gian thực để phát hiện các mẫu chuyển động bất thường. Một sự kiện té ngã thường được xác định khi gia tốc vượt ngưỡng (ví dụ: 3g) hoặc khi chuyển động đột ngột dừng lại, theo sau là trạng thái bất động~\cite{xu2023}. Các thuật toán phổ biến bao gồm ngưỡng tĩnh, học máy (SVM, Decision Tree) và học sâu (LSTM, CNN). Ví dụ,~\cite{hussain2019} sử dụng MPU6050 với LSTM, đạt độ chính xác 94.1\% trên tập dữ liệu SisFall.
    \item \textbf{Ưu điểm:} 
    \begin{itemize}
        \item Độ chính xác cao trong việc ghi nhận dữ liệu chuyển động cá nhân.
        \item Phản hồi nhanh, phù hợp với giám sát thời gian thực.
        \item Hoạt động độc lập với điều kiện môi trường như ánh sáng hay cấu trúc không gian.
        \item Chi phí thấp với các thiết bị như đồng hồ thông minh hoặc vòng đeo tay~\cite{wearable20152024}.
    \end{itemize}
    \item \textbf{Nhược điểm:}
    \begin{itemize}
        \item Yêu cầu người dùng đeo thiết bị liên tục, gây bất tiện hoặc dễ bị quên, đặc biệt với người cao tuổi.
        \item Dễ gây cảnh báo sai khi thực hiện các hoạt động mạnh như chạy, nhảy hoặc ngồi xuống nhanh, với tỷ lệ false positive có thể lên đến 20\%~\cite{alarifi2021}.
        \item Pin và hiệu chuẩn định kỳ là vấn đề, đặc biệt với các cảm biến cần bảo trì thường xuyên.
    \end{itemize}
\end{itemize}

% Subsection for environment-based sensor methods
\subsection{Phương pháp dựa trên cảm biến môi trường}

Phương pháp này sử dụng các cảm biến cố định trong không gian sống, phù hợp cho giám sát tại nhà hoặc cơ sở y tế mà không cần người dùng đeo thiết bị.

\begin{itemize}
    \item \textbf{Cơ chế hoạt động:} Các cảm biến phổ biến bao gồm \textbf{cảm biến áp suất sàn} (phát hiện thay đổi trọng lượng), \textbf{cảm biến hồng ngoại thụ động (PIR)} (phát hiện chuyển động hoặc bất động) và \textbf{cảm biến âm thanh} (nhận diện tiếng va chạm). Ví dụ, cảm biến áp suất sàn phát hiện sự thay đổi trọng lượng bất thường (ví dụ: từ 60 kg xuống 0 kg trong 0.5 giây), trong khi cảm biến PIR ghi nhận sự bất động kéo dài~\cite{smartfloor2024}. Các hệ thống hiện đại tích hợp AI để phân tích dữ liệu, như mô hình học sâu trên cảm biến âm thanh đạt độ chính xác 92\% trong môi trường thử nghiệm~\cite{chen2024}. 
    \item \textbf{Ưu điểm:} 
    \begin{itemize}
        \item Không xâm phạm, không yêu cầu người dùng đeo thiết bị, phù hợp với người cao tuổi không muốn sử dụng thiết bị cá nhân.
        \item Có thể giám sát nhiều người trong một khu vực, lý tưởng cho nhà dưỡng lão hoặc bệnh viện.
        \item Dễ tích hợp vào hệ thống nhà thông minh.
    \end{itemize}
    \item \textbf{Nhược điểm:}
    \begin{itemize}
        \item Chi phí lắp đặt cao, đặc biệt với cảm biến sàn, có thể lên đến hàng nghìn USD cho một căn hộ~\cite{smartfloor2024}.
        \item Phạm vi giám sát giới hạn, dẫn đến "điểm mù" ở các khu vực không có cảm biến.
        \item Khó phân biệt giữa người và vật thể (ví dụ: vật nuôi, đồ vật rơi), gây cảnh báo sai với tỷ lệ lên đến 15\% trong môi trường phức tạp~\cite{chen2024}.
    \end{itemize}
\end{itemize}

% Subsection for vision-based methods
\subsection{Phương pháp dựa trên thị giác máy tính}

Phương pháp này sử dụng camera và thuật toán xử lý ảnh để phân tích tư thế và chuyển động, mang lại thông tin trực quan phong phú mà không cần tiếp xúc vật lý.

\begin{itemize}
    \item \textbf{Cơ chế hoạt động:} Hệ thống sử dụng camera RGB, camera độ sâu (RGB-D) hoặc camera hồng ngoại để thu thập video. Các bước xử lý bao gồm: (1) phát hiện con người bằng mô hình như YOLOv5/v8, (2) ước tính tư thế với các framework như OpenPose, MediaPipe hoặc MoveNet, trích xuất các điểm khớp xương (keypoints), và (3) phân tích động học keypoints để xác định té ngã (ví dụ: chuyển từ tư thế đứng sang nằm trong dưới 1 giây). Ví dụ, Bugarin và cộng sự~\cite{bugarin2022} sử dụng MediaPipe đạt F1-score 91.4\% trên tập dữ liệu MCFD. Các mô hình học sâu như Transformer hoặc CNN-LSTM được áp dụng để cải thiện độ chính xác, đạt mAP 98.6\% trong một số nghiên cứu~\cite{han2024}.
    \item \textbf{Ưu điểm:} 
    \begin{itemize}
        \item Cung cấp thông tin bối cảnh trực quan, hỗ trợ xác nhận sự cố dễ dàng hơn thông qua hình ảnh hoặc video.
        \item Không yêu cầu người dùng đeo thiết bị, giảm bất tiện và tăng chấp nhận từ người dùng.
        \item Có thể tích hợp với hệ thống giám sát an ninh hiện có.
    \end{itemize}
    \item \textbf{Nhược điểm:}
    \begin{itemize}
        \item Lo ngại về quyền riêng tư, đặc biệt khi sử dụng camera trong không gian riêng như phòng ngủ hoặc phòng tắm.
        \item Hiệu suất phụ thuộc vào ánh sáng, góc quay và sự che khuất, với độ chính xác giảm 10--20\% trong điều kiện ánh sáng yếu~\cite{saraswat2024}.
        \item Yêu cầu phần cứng tính toán mạnh, đặc biệt với các mô hình học sâu như Transformer, gây khó khăn cho triển khai trên thiết bị biên~\cite{stylios2024}.
    \end{itemize}
\end{itemize}

% Subsection for multi-modal methods
\subsection{Phương pháp kết hợp đa phương thức}

Phương pháp này kết hợp nhiều nguồn dữ liệu để cải thiện độ chính xác và độ tin cậy, tận dụng ưu điểm của các phương pháp riêng lẻ.

\begin{itemize}
    \item \textbf{Cơ chế hoạt động:} Hệ thống tích hợp dữ liệu từ cảm biến đeo (IMU), camera (RGB hoặc RGB-D), và đôi khi cảm biến môi trường. Các thuật toán \textbf{sensor fusion} (như Kalman Filter hoặc học sâu) kết hợp dữ liệu để đưa ra quyết định. Ví dụ, một sự kiện té ngã được xác nhận khi cảm biến IMU ghi nhận gia tốc bất thường và camera phát hiện tư thế nằm ngang~\cite{rougier2011}. Keskes và Noumeir~\cite{keskes2021} sử dụng mạng ST-GCN để xử lý đồng thời dữ liệu skeleton từ OpenPose và tín hiệu IMU, đạt F1-score 93.2\%. Các nghiên cứu gần đây (2024) tích hợp thêm cảm biến mmWave (như MR60FDA2) để tăng độ chính xác lên 95\% trong môi trường phức tạp~\cite{mmwave2025}.
    \item \textbf{Ưu điểm:} 
    \begin{itemize}
        \item Giảm đáng kể tỷ lệ cảnh báo sai (xuống dưới 5\% trong một số hệ thống) nhờ xác minh đa nguồn~\cite{multimodal2024}.
        \item Mở rộng phạm vi giám sát, phù hợp cho cả trong nhà và ngoài trời.
        \item Tăng tính linh hoạt, có thể tùy chỉnh theo môi trường và nhu cầu cụ thể.
    \end{itemize}
    \item \textbf{Nhược điểm:}
    \begin{itemize}
        \item Độ phức tạp cao, yêu cầu thuật toán và phần cứng xử lý mạnh, tăng chi phí triển khai.
        \item Tiêu thụ năng lượng lớn, đặc biệt khi tích hợp nhiều cảm biến và camera.
        \item Khó khăn trong việc đồng bộ dữ liệu từ các nguồn khác nhau, có thể gây độ trễ trong xử lý thời gian thực~\cite{liu2018}.
    \end{itemize}
\end{itemize}

\subsection{Lợi thế của phương pháp tiếp cận kết hợp đa phương thức}
Các phương pháp phát hiện té ngã dựa trên một nguồn dữ liệu duy nhất (chẳng hạn như chỉ cảm biến quán tính hoặc chỉ thị giác máy tính) thường đối mặt với những hạn chế về độ chính xác và khả năng ứng dụng trong các bối cảnh đa dạng \cite{researchgate_hybrid}. Để khắc phục những điểm yếu này, hệ thống của đề tài đã phát triển một phương pháp tiếp cận đa phương thức độc đáo, tích hợp và tận dụng thế mạnh của cả hai công nghệ, mang lại nhiều lợi ích vượt trội.

\subsubsection{Cải thiện độ chính xác và giảm cảnh báo sai thông qua kết hợp dữ liệu}
Một trong những thách thức lớn nhất của các hệ thống phát hiện té ngã là tỷ lệ cảnh báo sai cao. Phương pháp tiếp cận đa phương thức giải quyết vấn đề này bằng cách sử dụng cơ chế \textbf{kết hợp dữ liệu (data fusion)} ở cấp độ quyết định, một kỹ thuật đã được chứng minh là tăng cường độ tin cậy của hệ thống \cite{mdpi_data_fusion}. Cụ thể, một sự kiện chỉ được xác nhận là té ngã khi cả hai hệ thống con đưa ra kết luận độc lập và khớp với nhau:

\begin{itemize}
    \item \textbf{Phát hiện từ cảm biến quán tính (IMU):} Thiết bị đeo trên người, tích hợp cảm biến gia tốc và con quay hồi chuyển, phân tích các thay đổi đột ngột về gia tốc tuyến tính và tốc độ góc \cite{resna_imu}. Thuật toán sẽ kích hoạt khi gia tốc vượt ngưỡng xác định, theo sau là một trạng thái bất động kéo dài.
    \item \textbf{Phát hiện từ thị giác máy tính:} Máy chủ cục bộ xử lý luồng video từ camera, sử dụng các thuật toán ước tính tư thế tiên tiến để xác định vị trí các khớp xương. Một mô hình logic sẽ nhận diện sự thay đổi tư thế đột ngột từ đứng thẳng sang nằm ngang, được biểu thị bằng tỷ lệ giữa chiều cao và chiều rộng cơ thể giảm xuống dưới một ngưỡng nhất định.
\end{itemize}

Bảng \ref{tab:so_sanh_phuong_phap} dưới đây minh họa rõ ràng ưu điểm của cơ chế kết hợp dữ liệu so với các phương pháp đơn lẻ.

\begin{table}[h!]
    \centering
    \caption{So sánh các phương pháp phát hiện té ngã}
    \label{tab:so_sanh_phuong_phap}
    \begin{tabular}{|p{0.25\linewidth}|p{0.3\linewidth}|p{0.3\linewidth}|}
        \hline
        \textbf{Kịch bản} & \textbf{Hệ thống đơn lẻ} & \textbf{Hệ thống đa phương thức} \\
        \hline
        Người dùng ngồi nhanh xuống ghế & Có khả năng cảnh báo sai do thay đổi gia tốc và tư thế đột ngột. & Không cảnh báo vì tư thế "nằm" trên sàn không được xác nhận. \\
        \hline
        Người dùng té ngã thật sự & Cả hai hệ thống đều có thể đưa ra cảnh báo chính xác. & Cảnh báo chỉ được kích hoạt khi cả hai tín hiệu xác nhận (thay đổi gia tốc/tư thế và trạng thái nằm ngang). \\
        \hline
        Người dùng té ngã ngoài tầm camera & Không thể phát hiện té ngã. & Vẫn phát hiện nhờ cảm biến IMU trên thiết bị đeo. \\
        \hline
    \end{tabular}
\end{table}

\subsubsection{Mở rộng phạm vi giám sát và tính linh hoạt}
Một trong những ưu điểm cốt lõi của phương pháp đa phương thức là khả năng mở rộng phạm vi giám sát, vượt qua các hạn chế vật lý của các giải pháp truyền thống. Hệ thống có khả năng chuyển đổi linh hoạt giữa hai chế độ hoạt động:

\begin{itemize}
    \item \textbf{Chế độ giám sát tại chỗ:} Khi người dùng ở trong tầm bao phủ của camera, hệ thống hoạt động ở chế độ này. Dữ liệu video được xử lý cục bộ trên máy chủ (on-premise processing), đảm bảo quyền riêng tư và tận dụng được sức mạnh tính toán của các mô hình học sâu.
    \item \textbf{Chế độ giám sát di động:} Khi người dùng di chuyển ra khỏi tầm nhìn của camera, hệ thống tự động chuyển sang chế độ di động. Thiết bị đeo sử dụng \textbf{xử lý tại thiết bị} để phân tích dữ liệu cảm biến và gửi cảnh báo cùng tọa độ vị trí qua mạng di động, đảm bảo an toàn không bị gián đoạn và giải quyết triệt để bài toán "điểm mù" của camera \cite{researchgate_edge_computing}.
\end{itemize}

\subsubsection{Tính ổn định, kinh tế và hiệu quả triển khai}
Việc tích hợp các công nghệ mã nguồn mở và kiến trúc module mang lại một giải pháp ổn định, kinh tế và dễ dàng triển khai.

\begin{itemize}
    \item \textbf{Tính dự phòng trong truyền thông:} Hệ thống được thiết kế với hai kênh truyền thông độc lập. Kênh mạng nội bộ (LAN) sử dụng giao thức \textbf{SIP} để gọi điện khẩn cấp, hoạt động độc lập với Internet. Kênh Internet sử dụng giao thức \textbf{MQTT} nhẹ và hiệu quả, lý tưởng cho việc giám sát từ xa trong môi trường IoT \cite{parangat_mqtt}. Sự kết hợp này đảm bảo cảnh báo luôn được gửi đi ngay cả khi một trong hai kênh gặp sự cố.
    \item \textbf{Hiệu quả kinh tế:} Bằng cách sử dụng các linh kiện nhúng giá thành thấp và các nền tảng phần mềm mã nguồn mở, tổng chi phí đầu tư ban đầu thấp hơn đáng kể so với các giải pháp thương mại, làm tăng tính khả thi của dự án.
    \item \textbf{Khả năng mở rộng:} Kiến trúc module cho phép dễ dàng mở rộng hệ thống bằng cách thêm các thiết bị camera và cảm biến mới, phù hợp cho việc triển khai tại các cơ sở y tế hoặc viện dưỡng lão.
\end{itemize}

\subsection{Bảo mật và quyền riêng tư}
Vấn đề bảo mật và quyền riêng tư là ưu tiên hàng đầu, đặc biệt với các hệ thống sử dụng camera. Hệ thống đã giải quyết vấn đề này thông qua các cơ chế sau:
\begin{itemize}
    \item \textbf{Xử lý cục bộ:} Dữ liệu video thô được xử lý trên máy chủ cục bộ và không bao giờ được truyền ra Internet, giảm thiểu tối đa nguy cơ bị lộ lọt.
    \item \textbf{Dữ liệu tối thiểu:} Chỉ các thông tin đã được xử lý (như sự kiện té ngã và tọa độ) được gửi đi.
    \item \textbf{Tùy chọn linh hoạt:} Hệ thống có thể được cấu hình để chỉ sử dụng cảm biến ở những khu vực nhạy cảm về quyền riêng tư như khu vực vệ sinh và sinh hoạt cá nhân.
\end{itemize}
