\section{Các Giao Thức Truyền Thông}
\label{sec:communication_protocols}

Phần này trình bày các giao thức truyền thông cơ bản cho phép trao đổi dữ liệu và điều khiển trong hệ thống cảnh báo viễn thông nhúng IoT. Các giao thức bao gồm Giao thức Khởi tạo Phiên (SIP), Message Queuing Telemetry Transport (MQTT), và JavaScript Object Notation (JSON).

\subsection{Giao Thức Khởi Tạo Phiên (SIP)}
\label{subsec:sip_protocol}

SIP là giao thức báo hiệu cốt lõi cho các hệ thống VoIP, dùng để khởi tạo, quản lý, và kết thúc các phiên đa phương tiện qua mạng IP~\cite{sip_rfc3261}. Nó hỗ trợ các tính năng như gọi thoại, hội nghị video, nhắn tin tức thời, và thông tin hiện diện.

\subsubsection{Kiến Trúc và Luồng Thông Điệp}
\label{subsubsec:sip_architecture}

SIP hoạt động bằng cách trao đổi thông điệp giữa User Agent Clients (UACs) và User Agent Servers (UASs), với sự hỗ trợ của máy chủ proxy và đăng ký. Các thông điệp chính bao gồm:

\begin{itemize}
    \item \textbf{INVITE}: Khởi tạo phiên hoặc cuộc gọi.
    \item \textbf{ACK}: Xác nhận phản hồi thành công cho INVITE.
    \item \textbf{BYE}: Kết thúc phiên.
    \item \textbf{CANCEL}: Hủy yêu cầu đang chờ.
    \item \textbf{REGISTER}: Đăng ký thông tin thiết bị.
    \item \textbf{OPTIONS}: Truy vấn khả năng của máy chủ.
\end{itemize}

SIP kết hợp với Session Description Protocol (SDP) để định nghĩa tham số phiên (codec, cổng) và Real-time Transport Protocol (RTP) để truyền dữ liệu thoại/video, đảm bảo định tuyến linh hoạt và tối ưu hóa mạng.

\subsubsection{SIP trong Hệ Thống Cảnh Báo Asterisk}
\label{subsubsec:sip_asterisk_integration}

Trong hệ thống cảnh báo dựa trên Asterisk, SIP kết nối các thiết bị như điện thoại IP, softphone, và PBX qua SIP trunk, mang lại các lợi ích:

\begin{itemize}
    \item \textbf{Quản lý tập trung}: Quản lý thống nhất thiết bị đầu cuối.
    \item \textbf{Tương thích cao}: Hỗ trợ đa dạng thiết bị và nền tảng.
    \item \textbf{Tuân thủ tiêu chuẩn}: Tích hợp dễ dàng với hạ tầng viễn thông.
\end{itemize}

\subsubsection{Luồng Phân Phối Cảnh Báo}
\label{subsubsec:sip_alert_flows}

Luồng cuộc gọi SIP cho cảnh báo tự động bao gồm các bước:

\paragraph{Đăng Ký}
Thiết bị SIP gửi yêu cầu REGISTER tới Asterisk (Registrar Server) để xác thực và lưu thông tin liên lạc.

\paragraph{Thiết Lập Phiên}
Khi phát hiện điều kiện nghiêm trọng, ứng dụng Linux ra lệnh cho Asterisk khởi tạo cuộc gọi qua Asterisk Manager Interface (AMI) bằng lệnh Originate. Asterisk gửi INVITE tới thiết bị đích, nhận phản hồi (180 Ringing, 200 OK), và xác nhận bằng ACK. Header \texttt{Alert-Info: ;info=alert-autoanswer} có thể được dùng để tự động trả lời.

\paragraph{Trao Đổi Dữ Liệu}
Sau khi thiết lập, thông điệp cảnh báo được truyền qua RTP. Với Text-to-Speech (TTS), Asterisk tạo âm thanh từ văn bản và phát tới người nhận.

\paragraph{Kết Thúc Phiên}
Yêu cầu BYE từ một bên kết thúc phiên, được xác nhận bằng 200 OK.

\subsubsection{Tích Hợp SMS qua SIP}
\label{subsubsec:sip_sms}

SIP hỗ trợ gửi SMS trong Asterisk bằng cấu hình \texttt{textsupport=yes} trong \texttt{sip.conf} và logic trong \texttt{extensions.conf}, cho phép gửi cảnh báo văn bản qua nhà cung cấp SIP hoặc gateway GSM.

\subsection{Giao Thức Message Queuing Telemetry Transport (MQTT)}
\label{subsec:mqtt_protocol}

MQTT là giao thức nhẹ, tối ưu cho truyền thông máy-với-máy (M2M) trong IoT~\cite{mqtt_oasis_standard}, phù hợp với thiết bị hạn chế tài nguyên.

\subsubsection{Kiến Trúc Publish/Subscribe}
\label{subsubsec:mqtt_pubsub}

MQTT sử dụng mô hình publish/subscribe, trong đó client gửi (publisher) và nhận (subscriber) thông điệp qua broker trung tâm, mang lại lợi ích:

\begin{itemize}
    \item \textbf{Tách rời không gian}: Không cần biết vị trí mạng.
    \item \textbf{Tách rời thời gian}: Không yêu cầu kết nối đồng thời.
    \item \textbf{Tách rời đồng bộ}: Truyền và nhận độc lập.
\end{itemize}

\subsubsection{Mức Quality of Service (QoS)}
\label{subsubsec:mqtt_qos}

MQTT cung cấp ba mức QoS:

\begin{itemize}
    \item \textbf{QoS 0}: Gửi một lần, không xác nhận, độ trễ thấp.
    \item \textbf{QoS 1}: Gửi ít nhất một lần, có xác nhận, có thể trùng lặp.
    \item \textbf{QoS 2}: Gửi chính xác một lần qua bắt tay bốn bước, độ tin cậy cao.
\end{itemize}

\subsubsection{MQTT trong IoT-Viễn Thông}
\label{subsubsec:mqtt_iot_integration}

MQTT hiệu quả cho IoT nhờ:

\begin{itemize}
    \item \textbf{Tiết kiệm tài nguyên}: Header nhỏ, yêu cầu thấp về bộ nhớ và băng thông.
    \item \textbf{Phục hồi mạng}: Hoạt động tốt trên mạng không ổn định.
    \item \textbf{Dữ liệu thời gian thực}: Hỗ trợ truyền cảm biến nhanh.
    \item \textbf{Giao tiếp hai chiều}: Cho phép gửi lệnh từ máy chủ tới thiết bị.
\end{itemize}

\subsection{JavaScript Object Notation (JSON)}
\label{subsec:json_format}

JSON là định dạng trao đổi dữ liệu nhẹ, dễ đọc, được dùng rộng rãi trong IoT và viễn thông~\cite{json_rfc7159}.

\subsubsection{Ứng Dụng JSON}
\label{subsubsec:json_applications}

\begin{itemize}
    \item \textbf{Cấu hình}: Lưu thông tin cho ESP32 (mạng, cảm biến) và máy chủ Linux (cơ sở dữ liệu, API).
    \item \textbf{Trao đổi dữ liệu}: Đóng gói số đọc cảm biến và cảnh báo giữa ESP32, máy chủ, và Asterisk.
    \item \textbf{Tương thích}: Hỗ trợ tích hợp đa nền tảng, dễ phân tích.
\end{itemize}

\subsubsection{Ví Dụ JSON}
\label{subsubsec:json_examples}

\paragraph{Cấu Hình ESP32}
\begin{lstlisting}[language=json, caption=Cấu hình ESP32, label=lst:esp32_config]
{
    "network": {
        "ssid": "IoT_Network",           // Tên mạng Wi-Fi
        "password": "MatKhauBaoMat123", // Mật khẩu Wi-Fi
        "mqtt_broker": "192.168.1.100", // Địa chỉ broker
        "mqtt_port": 1883               // Cổng MQTT
    },
    "device": {
        "id": "ESP32_Sensor_01",        // ID thiết bị
        "location": "Tang_May_A"        // Vị trí
    }
}
\end{lstlisting}

\paragraph{Dữ Liệu Cảm Biến}
\begin{lstlisting}[language=json, caption=Payload cảm biến MQTT, label=lst:sensor_payload]
{
    "device_id": "ESP32_Sensor_01", // ID thiết bị
    "timestamp": "2024-03-15T10:30:00Z",
    "readings": {
        "temperature": {
            "value": 87.5,          // Nhiệt độ
            "unit": "celsius",
            "status": "critical"
        }
    }
}
\end{lstlisting}

\paragraph{Cảnh Báo}
\begin{lstlisting}[language=json, caption=Payload cảnh báo, label=lst:alert_payload]
{
    "alert_id": "ALT_20240315_001", // ID cảnh báo
    "timestamp": "2024-03-15T10:30:01Z",
    "details": {
        "parameter": "temperature",
        "current_value": 87.5,
        "threshold_exceeded": 85.0
    },
    "telecom_actions": {
        "voice_call": {
            "recipients": ["doi_bao_tri"],
            "tts_message": "Cảnh báo: Nhiệt độ 87.5°C tại Tầng Máy A"
        }
    }
}
\end{lstlisting}

\subsection{Tích Hợp Giao Thức}
\label{subsec:protocol_integration}

SIP, MQTT, và JSON tạo thành framework truyền thông mạnh mẽ cho hệ thống cảnh báo IoT. Luồng dữ liệu bao gồm:

\begin{enumerate}
    \item \textbf{Thu thập dữ liệu}: ESP32 định dạng số đọc cảm biến thành JSON.
    \item \textbf{Xuất bản MQTT}: Gửi JSON qua topic MQTT với QoS phù hợp.
    \item \textbf{Xử lý máy chủ}: Máy chủ Linux phân tích JSON, đánh giá cảnh báo.
    \item \textbf{Kích hoạt viễn thông}: Dùng API Asterisk để gửi cảnh báo qua SIP.
    \item \textbf{Lưu trữ}: Ghi log dữ liệu và cảnh báo để kiểm toán.
\end{enumerate}

\subsection{Bảo Mật}
\label{subsec:protocol_security}

\begin{itemize}
    \item \textbf{SIP}: Sử dụng TLS (SIPS) và SRTP để mã hóa, xác thực ngăn truy cập trái phép.
    \item \textbf{MQTT}: Hỗ trợ TLS/SSL, xác thực tên người dùng/mật khẩu, và ACLs.
    \item \textbf{JSON}: Xác thực đầu vào, mã hóa dữ liệu nhạy cảm khi truyền và lưu trữ.
\end{itemize}

\subsection{Tóm Tắt}
\label{subsec:protocol_summary}

SIP, MQTT, và JSON tạo nên hệ thống truyền thông đáng tin cậy cho cảnh báo IoT. SIP hỗ trợ báo hiệu, MQTT truyền dữ liệu thời gian thực, và JSON đảm bảo trao đổi dữ liệu linh hoạt, tạo nền tảng an toàn và có thể mở rộng.
