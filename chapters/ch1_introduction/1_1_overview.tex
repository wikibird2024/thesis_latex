\section{Tổng quan vấn đề té ngã và hệ thống phát hiện té ngã}

Té ngã là một trong những nguyên nhân hàng đầu gây chấn thương nghiêm trọng và tử vong không chủ ý, đặc biệt ở nhóm dân số cao tuổi. Theo số liệu của Tổ chức Y tế Thế giới (\textbf{WHO}), trung bình mỗi năm có khoảng 646.000 người tử vong do té ngã, trong đó hơn 80\% xảy ra ở các quốc gia thu nhập trung bình và thấp. Người cao tuổi (trên 65 tuổi) là nhóm có nguy cơ cao nhất, với tỷ lệ té ngã trên 30\% mỗi năm và tăng lên tới 50\% ở nhóm trên 80 tuổi. \textbf{Mức độ nghiêm trọng của vấn đề được minh họa chi tiết trong Hình 1.1.}

\begin{figure}[h]
    \centering
    \begin{tikzpicture}
        \begin{axis}[
            ybar,
            ymin=0,
            ymax=60, % Đặt max y để biểu đồ cân đối
            ylabel={Tỷ lệ Té ngã (\%)},
            xlabel={Nhóm Tuổi},
            xtick=data,
            symbolic x coords={65-74, 75-84, >85},
            bar width=20pt,
            nodes near coords,
            nodes near coords style={font=\small, color=black, /pgf/number format/.cd, fixed, precision=0},
            title={Tỷ lệ Té ngã Thường Niên Theo Nhóm Tuổi},
            enlarge x limits=0.3,
            height=7cm,
            width=0.8\textwidth,
            ymajorgrids=true,
            grid style=dashed,
            tick label style={font=\small},
            title style={font=\bfseries, align=center},
            xlabel style={font=\small},
            ylabel style={font=\small}
        ]
        \addplot[fill=blue!60] coordinates {
            (65-74, 30)
            (75-84, 40)
            (>85, 50)
        };
        \end{axis}
    \end{tikzpicture}
    \caption{Tỷ lệ té ngã thường niên ở người cao tuổi (Dữ liệu thống kê).}
    \label{fig:ti_le_nga}
\end{figure}
Riêng tại Việt Nam, tình trạng già hóa dân số đang diễn ra nhanh chóng; số liệu từ Tổng cục Thống kê cho thấy tỷ lệ người trên 60 tuổi chiếm khoảng 12\% dân số vào năm 2020 và dự báo sẽ tăng gấp đôi vào năm 2049. Điều này đặt ra áp lực đáng kể cho hệ thống y tế, đặc biệt trong công tác chăm sóc và giám sát người cao tuổi tại nhà hoặc tại các cơ sở chăm sóc chuyên biệt.

Té ngã không chỉ gây ra tổn thương thể chất như gãy xương hông, chấn thương sọ não hay tổn thương cột sống, mà còn ảnh hưởng nghiêm trọng đến tâm lý người bệnh. Một khi đã từng té ngã, người cao tuổi thường phát triển hội chứng sợ té, dẫn tới giảm hoạt động, suy yếu chức năng vận động và kéo theo sự suy giảm chất lượng cuộc sống toàn diện. Nghiêm trọng hơn, nếu té ngã xảy ra khi nạn nhân ở một mình và không có khả năng cầu cứu, hậu quả có thể rất nghiêm trọng. Thời gian nằm bất động kéo dài có thể gây ra các biến chứng như hạ thân nhiệt, mất nước, hoại tử do loét tì đè, hoặc thậm chí tử vong do không được cấp cứu kịp thời.

Trong bối cảnh đó, các \textbf{hệ thống phát hiện và cảnh báo té ngã tự động} đang trở thành một xu hướng tất yếu trong lĩnh vực y tế công nghệ cao. Các hệ thống này có khả năng phát hiện hành vi té ngã dựa trên phân tích tín hiệu từ nhiều nguồn cảm biến khác nhau như gia tốc kế, con quay hồi chuyển (\textbf{gyroscope}), cảm biến áp suất, hoặc \textbf{phân tích hình ảnh thời gian thực}. Khi được thiết kế đúng cách, các hệ thống này không chỉ nhận diện chính xác sự cố té ngã mà còn có khả năng gửi cảnh báo nhanh chóng đến người thân, nhân viên y tế hoặc các cơ quan cứu trợ. Đặc biệt, việc tích hợp định vị \textbf{GPS} giúp cung cấp vị trí chính xác của nạn nhân, tối ưu hoá thời gian phản ứng và can thiệp. Việc xây dựng và tối ưu hóa các hệ thống này chính là trọng tâm của luận văn này, và sẽ được đặt trong bối cảnh nghiên cứu toàn cầu ở phần tiếp theo.
