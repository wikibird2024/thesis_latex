
\section{Tóm tắt Chương}
\label{sec:chapter1_conclusion}

Chương này đã cung cấp cái nhìn tổng quan và đặt nền tảng cho toàn bộ \TENLUANVAN. Nghiên cứu xác định rõ nhu cầu cấp thiết về một giải pháp giám sát sức khỏe chủ động, tập trung vào \textbf{hệ thống phát hiện té ngã tích hợp}, nhằm hỗ trợ chăm sóc người cao tuổi và bệnh nhân tại nhà.

Bằng cách đánh giá các công trình nghiên cứu hiện tại trong lĩnh vực giám sát dựa trên \textbf{IMU} và \textbf{Thị giác Máy tính (Computer Vision, CV)}, nghiên cứu nhận diện được một khoảng trống kỹ thuật: chưa có giải pháp tích hợp hiệu quả giữa các mô hình \textbf{Ước lượng Tư thế Người (Human Pose Estimation, HPE)} tiên tiến (như \textit{MediaPipe Pose}) với nền tảng phần cứng nhúng chi phí thấp (như \textit{ESP32}). Sự kết hợp này là chìa khóa để đạt được cả \textbf{độ chính xác cao} từ CV và \textbf{tính di động, tiết kiệm chi phí} từ IMU.

Các mục tiêu cụ thể của nghiên cứu đã được xác định: xây dựng một \textbf{hệ thống phát hiện té ngã đáng tin cậy}, \textbf{hiệu quả về chi phí} và có khả năng phân tích các sự kiện đa giai đoạn bằng cách tận dụng dữ liệu đa cảm biến đồng thời.

Việc hoàn tất Chương Giới thiệu đã làm rõ vấn đề, tính cấp thiết và phạm vi nghiên cứu. Chương tiếp theo, \textit{Cơ sở Lý thuyết và Nền tảng Công nghệ}, sẽ đi sâu vào các nguyên lý khoa học và kỹ thuật – bao gồm \textbf{Thị giác Máy tính}, \textbf{Ước lượng Tư thế Người} và \textbf{Hệ thống nhúng} – làm nền tảng cho việc thiết kế và triển khai giải pháp tích hợp của luận văn.
