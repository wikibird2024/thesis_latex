% Preparing the LaTeX document with necessary packages
\section{Tình hình nghiên cứu trong và ngoài nước}

% Subsection for research context
\subsection{Tổng quan bối cảnh nghiên cứu}
Sự gia tăng dân số cao tuổi toàn cầu đã đặt ra những thách thức lớn về y tế và an toàn cá nhân, đặc biệt liên quan đến tai nạn té ngã. Theo Tổ chức Y tế Thế giới (WHO), té ngã là nguyên nhân hàng đầu gây thương tích không cố ý ở người cao tuổi, chiếm khoảng 684.000 ca tử vong mỗi năm và hàng triệu ca chấn thương không gây tử vong khác~\cite{who2021}. Thực trạng này đã thúc đẩy cộng đồng nghiên cứu toàn cầu tập trung phát triển các hệ thống phát hiện té ngã tự động, với mục tiêu giảm thiểu thương tích, cải thiện chất lượng chăm sóc và khả năng phản ứng kịp thời.

% Listing main approaches to fall detection
Các phương pháp phát hiện té ngã hiện nay có thể được phân loại chính thành ba hướng:  
\begin{enumerate}
    \item \textbf{Phương pháp dựa trên thị giác máy tính (Vision-based)}: Sử dụng camera để thu thập dữ liệu hình ảnh/video, áp dụng các thuật toán nhận diện tư thế người và phân tích động học để xác định tình huống té ngã.  
    \item \textbf{Phương pháp dựa trên cảm biến đeo trên người (Wearable sensor-based)}: Sử dụng các cảm biến quán tính như IMU, gia tốc kế, con quay hồi chuyển gắn trên thiết bị đeo tay hoặc dây đeo để theo dõi chuyển động và phát hiện té ngã.  
    \item \textbf{Phương pháp kết hợp đa phương thức (Multi-modal)}: Tích hợp dữ liệu từ nhiều nguồn, ví dụ kết hợp camera và cảm biến IMU, nhằm tận dụng ưu điểm của từng phương pháp và giảm tỷ lệ cảnh báo sai.
\end{enumerate}

% Subsection for international research
\subsection{Nghiên cứu quốc tế}
Trên thế giới, phát hiện té ngã đã trở thành lĩnh vực nghiên cứu sôi động trong chăm sóc sức khỏe thông minh, với nhiều thành tựu đáng chú ý cả về độ chính xác, tốc độ phản hồi và khả năng triển khai thực tế. Các tiến bộ gần đây (2024--2025) nhấn mạnh việc sử dụng ML và IoT để giám sát thời gian thực, giảm false alarm lên đến 80\% ở một số hệ thống~\cite{international2024}, và tích hợp AI trong thiết bị đeo để dự đoán rủi ro.

% Subsubsection for vision-based methods
\subsubsection{Phương pháp dựa trên thị giác máy tính (Vision-based)}
Các hệ thống dựa trên camera tập trung vào kỹ thuật ước lượng tư thế người (human pose estimation), với các framework nổi bật như OpenPose, MediaPipe và MoveNet. 

\begin{itemize}
    \item \textbf{MediaPipe Pose Estimation}: Nghiên cứu~\cite{bugarin2022} phát triển hệ thống phát hiện té ngã thời gian thực trên thiết bị di động, sử dụng MediaPipe để phân tích các điểm khớp xương chính (keypoints) của cơ thể. Hệ thống đạt F1-score 91.4\% trên tập dữ liệu MCFD, tích hợp giám sát và cảnh báo qua IoT. Tuy nhiên, độ chính xác bị ảnh hưởng khi điều kiện ánh sáng yếu hoặc cơ thể bị che khuất. Tương tự~\cite{saraswat2024} cũng sử dụng MediaPipe để trích xuất đặc trưng từ chuỗi video, giúp giảm chi phí tính toán so với các phương pháp truyền thống.  
    \item \textbf{Kết hợp YOLO và Deep Learning}: Một số nghiên cứu tích hợp mô hình nhận dạng vật thể YOLO với Pose Estimation. nghiên cứu~\cite{han2024} kết hợp YOLOv5 cải tiến với MediaPipe, đạt mAP 98.6\%. nghiên cứu~\cite{chen2022} đạt 92.7\% nhờ tích hợp đặc trưng đa quy mô. Để triển khai trên thiết bị biên (edge devices), nhiều mô hình nhẹ hơn như LFD-YOLO~\cite{lfdyolo2025} và SDES-YOLO~\cite{sdesyolo2025} được phát triển, cân bằng giữa hiệu suất và khả năng tính toán. Các nghiên cứu mới (2024) sử dụng vision-based với ML để phân tích hiệu suất, thách thức và ràng buộc, đạt độ chính xác cao hơn trong môi trường chăm sóc sức khỏe~\cite{mlvision2024}.
    \item \textbf{Ứng dụng Transformer và Pose Estimation}: Các nghiên cứu gần đây~\cite{stylios2024, zhang2022, pmc2024} áp dụng kiến trúc Transformer kết hợp MediaPipe để học mối quan hệ phức tạp giữa các keypoints theo thời gian, giúp nâng cao độ chính xác. Phương pháp này yêu cầu phần cứng mạnh để xử lý, nhưng đạt hiệu suất nhận diện vượt trội trong các tập dữ liệu lớn và phức tạp. Công nghệ mới như sàn thông minh và AI giảm false alarm lên 35\%~\cite{smartfloor2024}.
\end{itemize}

% Subsubsection for wearable sensor-based methods
\subsubsection{Phương pháp dựa trên cảm biến đeo trên người (Wearable sensor-based)}
Các hệ thống wearable thường sử dụng IMU như MPU6050, được tích hợp trong đồng hồ thông minh, vòng đeo tay hoặc dây đeo cơ thể. 

\begin{itemize}
    \item ~\cite{xu2023} phát triển hệ thống kết hợp loa thông minh và IoT, xác minh tình trạng té ngã, giảm false alarm.  
    \item ~\cite{hussain2019} sử dụng MPU6050 kết hợp LSTM, đạt 94.1\%.  
    \item ~\cite{alarifi2021} sử dụng bộ ba cảm biến, đạt 93.5\%.  
\end{itemize}

Các hệ thống này vẫn gặp hạn chế khi người dùng thực hiện các hoạt động sinh hoạt nhanh, dẫn tới cảnh báo sai (false alarm). Các nghiên cứu hiện đại đã thử nghiệm các thuật toán học sâu nhẹ (lightweight deep learning) để giảm tải trên thiết bị di động. Từ 2015--2024, cảm biến đeo tăng độ chính xác cao với ML và DL, tập trung vào ngăn ngừa rủi ro~\cite{wearable20152024}. Sáng tạo 2025 bao gồm áo airbag kích hoạt trong 0.18 giây~\cite{airbag2025}, và mmWave sensors như MR60FDA2 cho phát hiện té~\cite{mmwave2025}.

% Subsubsection for multi-modal methods
\subsubsection{Phương pháp kết hợp đa phương thức (Multi-modal)}
Để khắc phục nhược điểm của từng phương pháp, nhiều nghiên cứu kết hợp dữ liệu từ camera và cảm biến IMU:

\begin{itemize}
    \item ~\cite{rougier2011} kết hợp camera và cảm biến gia tốc, đạt độ chính xác 95.2\% và giảm tỷ lệ false alarm 4.1\%.  
    \item ~\cite{liu2018} tích hợp IMU và camera RGB-D, đạt 95.6\%.  
    \item ~\cite{keskes2021} sử dụng mạng ST-GCN xử lý đồng thời dữ liệu skeleton từ OpenPose và tín hiệu IMU, đạt F1-score 93.2\%.  
\end{itemize}

Các hệ thống multi-modal hiện đại còn thử nghiệm các phương pháp sensor fusion phức tạp, adaptive threshold, và các mô hình học sâu để tối ưu độ chính xác, giảm độ trễ và tăng tính ổn định khi triển khai thực tế. Các tiến bộ 2024 kết hợp cảm biến đơn/đa, ML để phát hiện và cứu hộ, giảm ER visits 80\%~\cite{multimodal2024}.

% Subsection for domestic research
\subsection{Nghiên cứu trong nước}
Tại Việt Nam, các nghiên cứu chủ yếu được thực hiện tại các trường đại học, với các hệ thống thử nghiệm (proof-of-concept) sử dụng Arduino, ESP32 và cảm biến IMU. Độ chính xác thường đạt 75--85\%, và các cảnh báo có thể gửi qua SMS hoặc ứng dụng di động. Các nghiên cứu tập trung vào đánh giá nguy cơ té ngã ở bệnh viện và người cao tuổi, sử dụng thang đo Morse hoặc mô hình dự báo.

% Listing notable domestic studies
Một số nghiên cứu đáng chú ý:
\begin{itemize}
    \item ~\cite{nguyen2024} xây dựng mô hình dự báo nguy cơ té ngã tại bệnh viện TP.HCM, xác định 18 yếu tố liên quan qua hồi quy Logistic, độ chính xác cao qua Bootstrap.  
    \item ~\cite{tran2017} phát hiện té ngã bằng gia tốc kế và LSTM, đạt độ chính xác 93.9\%.  
    \item ~\cite{phan2022} khảo sát nguy cơ té ngã tại Bệnh viện Đa khoa vùng Tây Nguyên, tỷ lệ cao 43.58\%.  
    \item ~\cite{nguyen2023} nghiên cứu té ngã hậu COVID-19 ở người cao tuổi, tỷ lệ 10.8\%, liên quan đến nhẹ cân và yếu cơ.  
    \item ~\cite{trinh2023} khảo sát té ngã ngoại viện, tỷ lệ 21.6\%, yếu tố rủi ro như đái tháo đường.
\end{itemize}

% Highlighting limitations
Hạn chế chính: Các nghiên cứu chủ yếu mô tả cắt ngang, thiếu dữ liệu lớn và tích hợp đa phương thức. Độ chính xác thấp hơn quốc tế do hạn chế công nghệ và mẫu nghiên cứu nhỏ. Cần thêm nghiên cứu về AI và IoT để cải thiện hệ thống thực tế.
