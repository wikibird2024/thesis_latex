
\chapter*{TÓM TẮT}
\addcontentsline{toc}{chapter}{TÓM TẮT}

Té ngã là một trong những nguyên nhân chính gây chấn thương nghiêm trọng ở người cao tuổi và các đối tượng cần giám sát đặc biệt. Luận văn này trình bày thiết kế và triển khai một hệ thống phát hiện và cảnh báo té ngã thời gian thực, được xây dựng trên kiến trúc tích hợp và phân tán toàn diện.

Hệ thống gồm hai thành phần phát hiện chính, mỗi thành phần được tối ưu hóa cho một phạm vi giám sát cụ thể. Thành phần thứ nhất là hệ thống cảm biến đeo trên người (wearable system), sử dụng cảm biến quán tính (IMU) để theo dõi chuyển động và tư thế của người dùng cho giám sát ở phạm vi lớn và di động. Thành phần thứ hai là hệ thống thị giác máy tính có phạm vi giám sát đặt cố định, sử dụng thuật toán phân tích video để giám sát các khu vực trọng yếu trong môi trường, như phòng khách hoặc phòng ngủ. Khi một sự kiện té ngã được phát hiện bởi bất kỳ thành phần nào, thông tin sẽ được truyền đến nền tảng xử lý trung tâm.

Logic xử và điều khiển cho thiết bị điện toán biên được thực thi trên nền tảng vi điều khiển ESP32, tối ưu về hiệu năng và chi phí. Khi sự kiện té ngã được xác nhận, dữ liệu cảnh báo được mã hóa dưới định dạng JSON và truyền tới máy chủ thông qua giao thức MQTT, đảm bảo thông tin được gửi đi với độ trễ tối thiểu. Kiến trúc này cho phép hệ thống duy trì tính tin cậy ngay cả khi một trong hai thành phần gặp sự cố.

Hệ thống còn tích hợp giải pháp viễn thông tự động dựa trên Asterisk PBX và module 4G-GPS, cho phép tự động thực hiện các cuộc gọi khẩn cấp, gửi tin nhắn SMS chứa tọa độ GPS chính xác, và chia sẻ hình ảnh từ hiện trường qua các ứng dụng nhắn tin OTT, cung cấp phản ứng nhanh chóng và đa kênh.

Kết quả là một kiến trúc mô-đun, có khả năng mở rộng cao, dễ dàng tích hợp thêm các loại cảm biến hoặc phương thức cảnh báo trong tương lai. Hệ thống thể hiện sự kết hợp hài hòa giữa IoT nhúng, viễn thông tự động và phân tích lưu trữ dữ liệu, góp phần nâng cao chất lượng cuộc sống và đảm bảo an toàn cho cộng đồng.
