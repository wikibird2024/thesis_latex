\chapter{Phụ lục A: Mã nguồn hệ thống}

Phụ lục này cung cấp các liên kết tới mã nguồn đầy đủ của toàn bộ hệ thống phát hiện và cảnh báo té ngã. Mã nguồn được quản lý và phát triển trên nền tảng GitHub cá nhân của tác giả.

\section{Mã nguồn ESP32 (cảm biến đeo)}
Mã nguồn chương trình nhúng trên vi điều khiển ESP32, bao gồm các mô-đun cảm biến (MPU6050), xử lý logic té ngã, mô-đun SIM4G-GPS, giao tiếp MQTT và Wi-Fi. Đây là phần firmware cốt lõi của thiết bị đeo tay.

\begin{itemize}
    \item \textbf{Link truy cập:} \url{https://github.com/wikibird2024/mainproject.git}
\end{itemize}

\section{Mã nguồn Python (intergrate\_fall)}
Mã nguồn hệ thống trên máy chủ, chịu trách nhiệm xử lý dữ liệu từ camera (phát hiện té ngã bằng xử lý ảnh), nhận dữ liệu từ MQTT, quản lý cơ sở dữ liệu và gửi các cảnh báo đa kênh (Telegram, tin nhắn, cuộc gọi).

\begin{itemize}
    \item \textbf{Link truy cập:} \url{https://github.com/wikibird2024/intergrate_fall.git}
\end{itemize}

\vspace{1cm}
Toàn bộ mã nguồn trên do tác giả \TENTACGIA{} trực tiếp xây dựng và phát triển, được công bố nhằm phục vụ cho việc tham khảo, đánh giá và tái sử dụng trong các nghiên cứu tiếp theo.
