

%%%%%%%%%%%%%%%%%%%%%%%%%%%%%%%%%%%%%%%%%%%%%%%%%%%%%%%%%%%%%%%%%%%%%%%
%       VIETNAMESE LATEX THESIS PREAMBLE - XeLaTeX + Cormorant Garamond
%%%%%%%%%%%%%%%%%%%%%%%%%%%%%%%%%%%%%%%%%%%%%%%%%%%%%%%%%%%%%%%%%%%%%%%

% --- XeLaTeX required ---
\usepackage{fontspec}
\usepackage{pifont}
% --- Polyglossia for Vietnamese and English ---
\usepackage{polyglossia}
\setmainlanguage{vietnamese}
\setotherlanguage{english}

%%%%%%%%%%%%%%%%%%%%%%%%%%%%%%%%%%%%%%%%%%%%%%%%%%%%%%%%%%%%%%%%%%%%%%%%
%                        FONT CONFIGURATION
%%%%%%%%%%%%%%%%%%%%%%%%%%%%%%%%%%%%%%%%%%%%%%%%%%%%%%%%%%%%%%%%%%%%%%%%

\usepackage{unicode-math}

% Main serif font: TeX Gyre Termes (clone Times New Roman, rõ ràng, an toàn)
\setmainfont{TeX Gyre Termes}[
  Ligatures=TeX,
  Scale=1.05 % hơi lớn hơn chút, dễ đọc khi in
]

% Sans-serif: TeX Gyre Heros (clone Helvetica – rõ, mạnh mẽ, hợp tiêu đề)
\setsansfont{TeX Gyre Heros}[
  Ligatures=TeX,
  Scale=1.0
]

% Monospace: Fira Code (cho code block, hỗ trợ ligature đẹp)
\setmonofont{Fira Code}[
  Scale=0.9,
  Contextuals={Alternate} % bật ligature code
]

% Math font: Latin Modern Math (ổn định, rõ ràng)
\setmathfont{Latin Modern Math}

% ============================
% Micro-typography
% ============================
\usepackage[protrusion=true, expansion=true]{microtype}
\microtypesetup{spacing=true, kerning=true}

% ============================
% Line spacing & Paragraph
% ============================
\usepackage{setspace}
\onehalfspacing           % 1.5 line spacing – chuẩn luận văn
\usepackage{indentfirst}  % thụt đầu dòng đoạn đầu
\setlength{\parindent}{1.2em}
\setlength{\parskip}{0pt}
%%%%%%%%%%%%%%%%%%%%%%%%%%%%%%%%%%%%%%%%%%%%%%%%%%%%%%%%%%%%%%%%%%%%%%%%
%                    ADDITIONAL PACKAGES
%%%%%%%%%%%%%%%%%%%%%%%%%%%%%%%%%%%%%%%%%%%%%%%%%%%%%%%%%%%%%%%%%%%%%%%%
\usepackage{xcolor}
\usepackage{minted} % Cache để tránh shell-escape mỗi lần
\usepackage{graphicx}
\graphicspath{{figures/}{chapters/ch2_background/}{chapters/ch3_methodology/}}
\usepackage{calc}
\usepackage{url}
\usepackage{amsmath}
\usepackage{caption}
\usepackage{subcaption}
\usepackage{tabularx}

% For graphic plot
\usepackage{tikz}
\usetikzlibrary{positioning, calc,arrows.meta}
\usepackage{pgfplots}
\pgfplotsset{compat=1.18}

% Caption style: centered, nhỏ hơn body text, bold label
\captionsetup{justification=centering, labelfont=bf, font=small}
\captionsetup[subfigure]{justification=centering}
\captionsetup{justification=centering, labelfont=bf}
\captionsetup[subfigure]{justification=centering}

% Màu cho minted
\definecolor{darkblue}{RGB}{0,0,139}
\definecolor{lightgray}{gray}{0.95}

% Cấu hình minted
\setminted{
    frame=single,
    linenos=true,
    numbersep=10pt,
    bgcolor=lightgray,
    fontsize=\small,
    breaklines=true,
    tabsize=4,
    xleftmargin=10pt,
    escapeinside=||
}

% Bibliography
\usepackage[backend=biber,style=ieee]{biblatex}
\addbibresource{references.bib}

%%%%%%%%%%%%%%%%%%%%%%%%%%%%%%%%%%%%%%%%%%%%%%%%%%%%%%%%%%%%%%%%%%%%%%%%
%                    CUSTOM MACROS FOR THESIS
%%%%%%%%%%%%%%%%%%%%%%%%%%%%%%%%%%%%%%%%%%%%%%%%%%%%%%%%%%%%%%%%%%%%%%%%
\newcommand{\TENLUANVAN}{Hệ thống phát hiện và cảnh báo té ngã thời gian thực tích hợp cảm biến, xử lý ảnh và định vị}
\newcommand{\THESISNAME}{Real-time fall detection and alert system integrating sensors, image processing, and positioning}
\newcommand{\TENTACGIA}{Trần Đức Hảo}
\newcommand{\MASOSV}{1734011}
\newcommand{\KHOA}{Điện - Điện tử}
\newcommand{\BOMON}{Viễn thông}
\newcommand{\TENNGUOIHUONGDAN}{PSG.TS Hà Hoàng Kha}
\newcommand{\NAMBAOVE}{2025}
\newcommand{\DEPARTMENT}{\KHOA}
\newcommand{\TENTACGIAFACULTY}{\TENTACGIA}
\newcommand{\MSSV}{\MASOSV}
\newcommand{\TENCANBO}{\TENNGUOIHUONGDAN}

%%%%%%%%%%%%%%%%%%%%%%%%%%%%%%%%%%%%%%%%%%%%%%%%%%%%%%%%%%%%%%%%%%%%%%%%
%                        PAGE & LAYOUT SETTINGS
%%%%%%%%%%%%%%%%%%%%%%%%%%%%%%%%%%%%%%%%%%%%%%%%%%%%%%%%%%%%%%%%%%%%%%%%
\usepackage{geometry}
\geometry{
    a4paper,
    top=2.5cm,
    bottom=2.5cm,
    left=3.0cm,
    right=2.0cm,
    headheight=18pt,
    headsep=12pt,
    footskip=20pt
}

\onehalfspacing
\usepackage{indentfirst}
\setlength{\parindent}{1.2em}
\setlength{\parskip}{0pt}
\tolerance=2000
\emergencystretch=0.5em
\hyphenpenalty=5000
\exhyphenpenalty=5000

%%%%%%%%%%%%%%%%%%%%%%%%%%%%%%%%%%%%%%%%%%%%%%%%%%%%%%%%%%%%%%%%%%%%%%%%
%                     HEADER, FOOTER & TITLES
%%%%%%%%%%%%%%%%%%%%%%%%%%%%%%%%%%%%%%%%%%%%%%%%%%%%%%%%%%%%%%%%%%%%%%%%
\usepackage{titlesec}
\usepackage{fancyhdr}
\pagestyle{fancy}
\fancyhf{}
\fancyhead[L]{\nouppercase{\leftmark}} % Set the chapter title on the left
\fancyhead[R]{\TENTACGIA}              % Set the author's name on the right
\fancyfoot[C]{\thepage}                % Set the page number in the center
\renewcommand{\headrulewidth}{0.4pt}
\renewcommand{\footrulewidth}{0pt}
\renewcommand{\chaptermark}[1]{\markboth{\chaptername\ \thechapter.\ #1}{}}

% Hyperref (load last)
\usepackage{hyperref}
\hypersetup{
    colorlinks=true,
    linkcolor=darkblue,
    urlcolor=darkblue,
    citecolor=darkblue,
    pdfauthor={\TENTACGIA},
    pdftitle={\TENLUANVAN}
}
